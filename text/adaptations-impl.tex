
In this section we will cover the implementation details that are relevant for the adaptations described in the previous section.

\subsection{Allocator Versions}

The reference version of the allocator has been abstracted into a base class named \texttt{TLSFBase}. From this base case, the user must provide a set of configuration variables that define how the allocator behaves. As such, each unique set of configuration variables define a new allocator. Additionally, an allocator may override the implementation of the methods defined in the base class so that computations may be done differently. This allows different versions to apply different methods for allocation and freeing for example, as well as applying different strategies for concurrency.

\subsection{New Bitmap Design}

Summarizing the insight from Section~\ref{sec:adaptations:reduced_allocation_range} regarding the number of required first- and second-levels, desire to use a single 64-bit bitmap and large-list, we can construct a new bitmap representation, as shown in Figure~\ref{fig:bitmap_flattening}. The new bitmap disregards the literal notion of ``two-level'' from Two-Level Segregated Fit and flattens the first- and second-level bitmaps to a single bitmap. The mapping for the bitmap is calculated by combining the first- and second-level mappings, which are calculated the same way as in the reference design. Hence, the mapping for the new bitmap is calculated using the formula below, where $f$ and $s$ are the first- and second-level indexes respectively.
\[
    I(f, s) = 4f + s
\]

Like is done in the reference design, the new bitmap places the classes of the lowest size at the least significant bits to make searching for the next non-empty free-list efficient using the find-first-set (\texttt{ffs}) bit instruction. Furthermore, the new design only requires searching a single bitmap, making a single \texttt{ffs} instructions enough to find free-lists in any first-level, unlike the reference design which requires two \texttt{ffs} instructions.

\begin{figure}[H]
    \centering
    \includesvg[width=0.8\textwidth]{figures/bitmap_flattening.svg}
    \vspace*{4mm}
    \caption{Flattening of the 2D-matrix representation of TLSF bitmaps into a single 64-bit value. The first-level bitmap is disregarded in favor of indexing the new flattened bitmap using the first-level value instead. The number of first-levels are 14, indicated by bits of the same color belonging to the same first-level. The number of second-levels are 4, as indicated by the same number of colored bits.}
    \label{fig:bitmap_flattening}
\end{figure}

The correlation between the new bitmap and free-lists is depicted in Figure~\ref{fig:bitmap_relationship}, adhering closely to the original TLSF design. The highest granularity of free-list size-ranges are found for the smallest allocations, which aligns well with most allocations being small inside most Java programs. This leads to less memory potentially being wasted and less splitting done since more blocks match the request size, thus saving performance.

\begin{figure}[H]
    \centering
    \vspace*{0.2cm}
    \includesvg[width=1.0\textwidth]{figures/bitmap_relationship.svg}
    \vspace*{1mm}
    \caption{Relationship between the new bitmap representation and accessing the corresponding free-lists.}
    \label{fig:bitmap_relationship}
\end{figure}

\subsection{Block Header Adjustments and 0-byte Header}
\label{sec:adaptations_impl:0-byte-header}

The block header, as designed in the reference implementation, is shown in Figure~\ref{fig:blockheader_adap_reference}. Here, the size and previous physical pointer (prev\_phys) are constant and the next and prev pointers are only used in the unused part of free blocks, and unused for allocated blocks. In contrast, Figure~\ref{fig:blockheader_adap_general} shows the adapted block header for the general version of the allocator. In this version, all four fields are used for both free and allocated blocks since the previous physical pointer has been rearranged to the end of the header. Rearranging the previous physical pointer to be the last field in the header, as shown in Figure~\ref{fig:blockheader_adap_optimized}, makes it possible to have the same definition for the general and optimized version, while the optimized version only uses the first 16 bytes of it.

\begin{figure}[H]
    \centering
    \begin{subfigure}[b]{0.3\textwidth}
        \centering
        \includesvg[width=\textwidth]{figures/blockheader_adap_reference.svg}
        \caption{Reference implementation block header.}
        \label{fig:blockheader_adap_reference}
    \end{subfigure}%
    \hfill
    \begin{subfigure}[b]{0.3\textwidth}
        \centering
        \includesvg[width=\textwidth]{figures/blockheader_adap_general.svg}
        \caption{Adapted general block header.}
        \label{fig:blockheader_adap_general}
    \end{subfigure}%
    \hfill
    \begin{subfigure}[b]{0.3\textwidth}
        \centering
        \includesvg[width=\textwidth]{figures/blockheader_adap_optimized.svg}
        \caption{Adapted optimized block header.}
        \label{fig:blockheader_adap_optimized}
    \end{subfigure}
    \caption{Overview of block header contents. Striped fields are only present in the unused part of free blocks and crossed out fields are unused.}
    \label{fig:blockheader_adaptations}
\end{figure}

As stated in Section~\ref{sec:adaptations:block-header-adjustments}, the size field can be ignored for allocated blocks if the block size is kept track of somewhere else and is provided upon calling \texttt{free()}. The benefit of doing this is that more memory is available for object allocation inside ZGC pages. Consequently, this means that the size must be fetched and perhaps computed from elsewhere, adding extra operations which might decrease performance. 

Additionally, to further minimize footprint, the next and prev pointers have been converted to offsets in the optimized version, allowing them to be 32 bits each. The conversion to and from offsets and pointers does mean adding extra operations as well, which might decrease performance further. Both the size and next/prev fields are unused for allocated blocks and stored in the unused parts for free blocks. In the optimized version, this results in the header taking up 0 bytes for allocated blocks, hence the name \textit{0-byte header}, and 16 bytes for free blocks.

In summary however, the increase of available memory when applying the block header optimization at the cost of potentially decreased performance, is most likely a trade-off worth making. This effect is further amplified when most allocations are small, which results in the size of the block header taking up proportionally more memory. However, this has to be further evaluated to draw conclusions, but in this case we optimize for potential memory efficiency over performance.

\subsection{Concurrency Implementation}
\label{sec:adaptations_impl:concurrency}

Implementing concurrency for the allocator will only require concurrent operations on its free-lists, as it is where blocks are stored and managed. In the reference design, the free-lists are designed as doubly-linked lists. However, lock-free operations on a doubly-linked-list will not be considered due to the complex operations required to support it. Instead, we will simplify the appearance of the free-lists with regard to previously mentioned adaptations to reduce their complexity.

By disregarding the pointer to the previous free block inside block headers, the appearance of the free-lists can be transformed from a doubly-linked list to a singly-linked list. This transformation is possible if we limit ourselves to deferred coalescing, since the previous pointer is only used when inserting blocks at positions other than the head of a free-list. Consequently, by only allowing updates at the head of a free-list, the singly-linked list effectively becomes a stack, simplifying the implementation of a lock-free solution.

M. Herlihy and N. Shavit~\cite[Chapter 11]{artofmpprogramming} describes the design and inner-workings of a lock-free stack from which the concurrent implementation of the allocator is largely based on. For reference, we will go over how a lock-free stack work and how it connects to the data structure inside the allocator.

A stack only requires two operations: \texttt{push()} and \texttt{pop()}, which are named \texttt{insert()} and \texttt{remove()} in the allocator to stay consistent with their meaning from the general version. The insert operation replaces the head of the stack, or free-list, with a new block, whilst the remove operation removes the head of the free-list and replaces it with the next block, if any, in the free-list. In order to make the allocator work in a thread-safe manner, we only need to make these two operations concurrent, since all other operations in the allocator can be done independently of other threads.

To be able to operate on the free-list concurrently without locks, we will use the atomic operations \texttt{load()}\footnote{\url{https://en.cppreference.com/w/cpp/atomic/atomic/load}} and \texttt{compare\_exchange()}\footnote{\url{https://en.cppreference.com/w/cpp/atomic/atomic/compare_exchange}}, provided in the C++ standard library. \texttt{load()} returns the current value of an atomic variable and \texttt{compare\_exchange()} compares the value of a variable with an expected value, and if they are equal, it is replaced by another value.

\subsubsection{Solving The ABA Problem}
\label{sec:adaptations_impl:aba_problem}

The ABA problem occurs when a thread employs compare-and-exchange to read a value it intends to change, gets interrupted, and upon resuming, finds the value it initially read remains unchanged, despite potential modifications by other threads. This can lead to successfully performing a compare-and-exchange operation even when modifications during the interruption could have occurred.

A strategy for solving the ABA problems is ensuring that all values, or pointers in the allocator, are inherently unique and used only once. This approach enables the detection of changes during thread preemption. However, pointers to blocks are not unique in the allocator and can be inserted and removed any number of times. One way to achieve pointer uniqueness is to apply a so-called version tag to each pointer, which makes it possible to uniquely identify pointers.

Version tagging is described by D. Dechev et al.~\cite{bjarne_aba} as a common and effective solution for the ABA problem. However, due to how compare-and-exchange work, it is desirable to store the pointer and version tag in a single word, i.e. 8-bytes on a 64-bit system. When using the allocator for pages in ZGC, this is not a problem as we can apply the same principle as in Section~\ref{sec:adaptations_impl:0-byte-header} to convert 64-bit pointers to 32-bit offsets. This conversion allows storing the version tag in the remaining 32-bits of the word, as illustrated in Figure~\ref{fig:concurrent_head_bits}.

\begin{figure}[H]
    \centering
    \vspace*{4mm}
    \includesvg[width=1\textwidth]{figures/concurrent_head_bits.svg}
    \caption{Bit representation of a free-list head in the allocator. The 32 most significant bits are used to store the offset and the 32 least significant bits are used to store the version tag.}
    \label{fig:concurrent_head_bits}
\end{figure}

\subsubsection{Insertion and removal of blocks}

The process of concurrently inserting a block into a free-list is outlined in Listing~\ref{algorithm:concurrent_insert_block}. Between loading the old head from the free-list and executing the \texttt{compare\_exchange()} operation, the new head must be constructed using data from the old head. However, there is a potential issue wherein the thread responsible for construction may be preempted, allowing another thread to modify the head during this time. To address this, if the \texttt{compare\_exchange()} operation fails, the thread will loop and create a new head until it is successfully performed.

\begin{lstlisting}[language=C++, caption={Concurrent insertion of a lock into the head of a free-list.}, label={algorithm:concurrent_insert_block}]
void insert_block(BlockHeader *blk) {
    Mapping mapping = get_mapping(blk->size);

    do {
        TaggedBlockHeader *head = blocks[mapping].load();
        BlockHeader *head_ptr = extract_ptr(head);
        uint32_t version = extract_version(head);
        blk_set_next(blk, head_ptr);
        TaggedBlockHeader *new_head = construct(blk, version + 1);
    } while(blocks[mapping].compare_exchange(head, new_head));

    update_bitmap(mapping);
}
\end{lstlisting}

The process of concurrently removing a block from a free-list, as illustrated in Listing~\ref{algorithm:concurrent_remove_block}, is similarly defined to block insertion. If the \texttt{compare\_exchange()} operation fails when removing a block, there is no guarantee that the same free-list contain additional blocks. In cases where thread preemption occurs during removal, other threads might exhaust the free-list, which will make further operations fail. To solve this, \texttt{remove\_block()} either succeeds or fails, without looping until successful like \texttt{insert\_block()} does. Consequently, the caller must recalculate the mapping and call \texttt{remove\_block()} again, and thus loop outside the \texttt{remove\_block()} function.

\begin{lstlisting}[language=C++, caption={Concurrent removal of the head of a free-list.}, label={algorithm:concurrent_remove_block}]
BlockHeader *remove_block(Mapping mapping) {
    TaggedBlockHeader *head = blocks[mapping].load();
    BlockHeader *head_ptr = extract_ptr(head);
    uint32_t version = extract_version(head);
    TaggedBlockHeader *new_head = construct(head->next, version + 1);

    if(blocks[mapping].compare_exchange(head, new_head)) {
        update_bitmap(mapping);
        return new_head;
    }

    return nullptr;
}
\end{lstlisting}

% However, for single-threaded use-cases, the worst-case is still bounded, as indicated by the results in Section~\ref{sec:allocation-performance}.

%%% Local Variables:
%%% mode: latex
%%% TeX-master: "main"
%%% End:

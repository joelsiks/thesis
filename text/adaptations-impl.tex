
This section provides the implementation specifics for the adaptations described in the previous section, focusing on those that require detailed explanation. It covers the technical aspects and modifications necessary to optimize the TLSF allocator for use with ZGC.

\subsection{Allocator Configurations}

The reference implementation of the allocator has been abstracted into a base class named \texttt{TLSFBase}. From the base class, the user must provide a set of configuration variables that define how the allocator behaves, where each unique set of configuration variables defines a new allocator. An allocator may override the implementation of the methods defined in the base class so that computations may be done differently. For example, this allows different versions to apply different allocation, freeing, and concurrency methods. The configuration variables are:

\begin{enumerate}
    \item Number of first- and second-levels (Sets the size range granularity and number of free-list).
    \item Minimum block size (The minimum size of blocks).
    \item Block header size (The size of the block header. Depends on what fields/bytes are used in the header).
    \item Usage of second-levels (When not using second-levels, the new bitmap design is used, as described in Section~\ref{sec:adaptations_impl:bitmap_design}, otherwise the reference implementation is used).
    \item Usage of deferred coalescing (Deferred coalescing disables immediate coalescing. Coalescing can only be done through explicit coalescing).
\end{enumerate}

\subsection{New Bitmap Design}
\label{sec:adaptations_impl:bitmap_design}

Summarizing the insight from Section~\ref{sec:adaptations:reduced_allocation_range} regarding the number of required first- and second-levels, the desire to use a single 64-bit bitmap and large-list, a new representation for the bitmap is constructed, as shown in Figure~\ref{fig:bitmap_flattening}. The new bitmap disregards the literal notion of ``two-level'' from Two-Level Segregated Fit and flattens the first- and second-level bitmaps to a single bitmap. The mapping for the bitmap is calculated by combining the first- and second-level mappings, which are calculated the same way as in the reference design. Hence, the mapping for the new bitmap is calculated using the formula below, where $f$ and $s$ are the first- and second-level indexes, respectively.
\[
    I(f, s) = 4f + s
\]

Like is done in the reference design, the new bitmap places the classes of the lowest size at the least significant bits to make searching for the next non-empty free-list efficient using the find-first-set (\texttt{ffs}) bit instruction. Furthermore, the new design only requires searching a single bitmap, making a single \texttt{ffs} instruction enough to find free-lists in any first-level, unlike the reference design, which requires two \texttt{ffs} instructions.

\begin{figure}[h]
    \centering
    \includesvg[width=0.8\textwidth]{figures/bitmap_flattening.svg}
    \vspace*{4mm}
    \caption{Flattening of the 2D-matrix representation of TLSF bitmaps into a single 64-bit value. The first-level bitmap is disregarded in favor of indexing the new flattened bitmap using the first-level value instead. The number of first-levels is 14, indicated by bits of the same color belonging to the same first-level. The number of second-levels is 4, as indicated by the same number of colored bits.}
    \label{fig:bitmap_flattening}
\end{figure}

The correlation between the new bitmap and free-lists is depicted in Figure~\ref{fig:bitmap_relationship}, adhering closely to the original TLSF design. The highest granularity of free-list size ranges is found for the smallest allocations, which aligns well with most allocations being small inside most Java programs. This leads to less memory potentially being wasted and less splitting done since more blocks match the request size, thus saving performance.

\begin{figure}[h]
    \centering
    \vspace*{0.2cm}
    \includesvg[width=1.0\textwidth]{figures/bitmap_relationship.svg}
    \vspace*{1mm}
    \caption{Relationship between the new bitmap representation and accessing the corresponding free-lists.}
    \label{fig:bitmap_relationship}
\end{figure}

\subsection{Block Header Adjustments and 0-byte Header}
\label{sec:adaptations_impl:0-byte-header}

The block header, as designed in the reference implementation, is shown in Figure~\ref{fig:blockheader_adap_reference}. Here, the size and previous physical pointer (prev\_phys) are constant, and the next and prev pointers are only used in the unused part of free blocks and unused for allocated blocks. In contrast, Figure~\ref{fig:blockheader_adap_general} shows the adapted block header for the general version of the allocator. In this version, all four fields are used for both free and allocated blocks since the previous physical pointer has been rearranged to the end of the header. Rearranging the previous physical pointer to be the last field in the header, as shown in Figure~\ref{fig:blockheader_adap_optimized}, makes it possible to have the same definition for the general and optimized versions, while the optimized version only uses the first 16 bytes of it.

\begin{figure}[h]
    \centering
    \begin{subfigure}[b]{0.3\textwidth}
        \centering
        \includesvg[width=\textwidth]{figures/blockheader_adap_reference.svg}
        \caption{Reference implementation block header.}
        \label{fig:blockheader_adap_reference}
    \end{subfigure}%
    \hfill
    \begin{subfigure}[b]{0.3\textwidth}
        \centering
        \includesvg[width=\textwidth]{figures/blockheader_adap_general.svg}
        \caption{Adapted general block header.}
        \label{fig:blockheader_adap_general}
    \end{subfigure}%
    \hfill
    \begin{subfigure}[b]{0.3\textwidth}
        \centering
        \includesvg[width=\textwidth]{figures/blockheader_adap_optimized.svg}
        \caption{Adapted optimized block header.}
        \label{fig:blockheader_adap_optimized}
    \end{subfigure}
    \caption{Overview of block header contents. Striped fields are only present in the unused part of free blocks and crossed-out fields are unused.}
    \label{fig:blockheader_adaptations}
\end{figure}

\newpage

As stated in Section~\ref{sec:adaptations:block-header-adjustments}, the size field can be ignored for allocated blocks if the block size is kept track of somewhere else and is provided upon calling \texttt{free()}. The benefit of doing this is that more memory is available for object allocation inside ZGC pages. Consequently, this means that the size must be fetched and perhaps computed from elsewhere, adding extra operations that might decrease performance. 

Additionally, to further minimize footprint, the next and prev pointers have been converted to offsets in the optimized version, allowing them to be 32 bits each. The conversion to and from offsets and pointers means adding extra operations, which might decrease performance further at the cost of memory efficiency. The next and prev fields are only used inside the unused memory of free blocks. Neither field is used for allocated blocks, which, in addition to not storing the size of the block, requires no header at all, hence the name \textit{0-byte header}.

In summary, the increase in available memory when applying block header optimization is at the cost of potentially decreased performance and is most likely a trade-off worth making. This is because most allocations in Java are small, which results in the size of the block header taking up proportionally more memory than for larger allocations.

\subsection{Implementing Concurrency Support}
\label{sec:adaptations_impl:concurrency}

Implementing concurrency support for the allocator will require being able to concurrently operate on its free-lists. The free-lists, in combination with the bitmaps that store data about them, are the critical sections of the allocator. In the reference design, the free-lists are designed as doubly-linked lists. However, lock-free operations on a doubly-linked list were not considered due to the complex operations required to support them. Instead, the appearance of the free-lists is simplified to make it easier to support concurrent operations.

By disregarding the pointer to the previous free block inside block headers, the appearance of the free-lists can be transformed from a doubly-linked list to a singly-linked list. This transformation is possible when not coalescing immediately, since the previous pointer is only used when inserting and removing blocks at positions other than the head of a free-list. Consequently, by only allowing updates at the head of a free-list, the singly-linked list effectively becomes a stack, simplifying the implementation of a lock-free solution.

Herlihy and Shavit~\cite{artofmpprogramming} describe the design and inner workings of the lock-free stack by Treiber~\cite{treiber}, on which the concurrent implementation of the allocator is largely based. For reference, a general overview of how a lock-free stack works and how it connects to the free-lists inside the allocator is described.

A stack only requires the push and pop operations, which are named \texttt{insert()} and \texttt{remove()} in the allocator to stay consistent with their meaning from the general version. The insert operation replaces the head of the stack, or free-list, with a new block, while the remove operation removes the head of the free-list and replaces it with the next block, if any, in the free-list. In order to make the allocator work in a thread-safe manner, only these two operations need to be concurrent, since all other operations in the allocator can be done independently of other threads.

The free-lists will be operated on concurrently without locks using the atomic operations \texttt{load()}\footnote{\url{https://en.cppreference.com/w/cpp/atomic/atomic/load}} and \texttt{compare\_exchange()}\footnote{\url{https://en.cppreference.com/w/cpp/atomic/atomic/compare_exchange}}, provided in the C++ standard library. \texttt{load()} returns the current value of an atomic variable, and \texttt{compare\_exchange()} compares the value of a variable with an expected value, and if they are equal, it is replaced by another value.

\subsubsection{Solving The ABA Problem}
\label{sec:adaptations_impl:aba_problem}

The ABA problem occurs when a thread employs compare-and-exchange to read a value it intends to change, gets interrupted, and, upon resuming, finds the value it initially read remains unchanged, despite potential modifications by other threads. Consider the following example:

\begin{enumerate}
    \item Thread 1 reads the value of a shared variable X as A and plans to change it to B.
    \item Thread 1 is interrupted before making the change.
    \item Thread 2 changes X from A to B and then back to A.
    \item Thread 1 resumes, sees X still as A, and successfully changes it to B, unaware of Thread 2's modifications.
\end{enumerate}

This scenario shows how Thread 1 is misled by the apparent unchanged value, which can lead to successfully performing the compare-and-exchange operation even when modifications during the interruption have occurred.

A strategy for solving the ABA problems is to ensure that all values, or pointers in the allocator, are inherently unique and used only once. This approach enables the detection of changes during thread preemption. However, pointers to blocks are not unique in the allocator and can be inserted and removed any number of times. One way to achieve pointer uniqueness is to apply a so-called version tag to each pointer, which makes it possible to uniquely identify pointers.

Version tagging is described by Dechev et al.~\cite{bjarne_aba} as a common and effective solution for the ABA problem. However, due to how compare-and-exchange works, it is desirable to store the pointer and version tag in a single word, i.e., 8 bytes on a 64-bit system. 

When using the allocator for pages in ZGC, this is not a problem, as the same principles applied in Section~\ref{sec:adaptations_impl:0-byte-header} can be used to convert 64-bit pointers to 32-bit offsets. This conversion allows storing the version tag in the remaining 32 bits of the word, as illustrated in Figure~\ref{fig:concurrent_head_bits}.

\begin{figure}[h]
    \centering
    \vspace*{4mm}
    \includesvg[width=1\textwidth]{figures/concurrent_head_bits.svg}
    \caption{Bit representation of a free-list head in the allocator. The 32 most significant bits are used to store the offset and the 32 least significant bits are used to store the version tag.}
    \label{fig:concurrent_head_bits}
\end{figure}

\subsubsection{Insertion and Removal of Blocks}

The process of concurrently inserting a block into a free-list is outlined in Listing~\ref{algorithm:concurrent_insert_block}. Between loading the old head from the free-list and executing the \texttt{compare\_exchange()} operation, the new head must be constructed using data from the old head. However, there is a potential issue wherein the thread responsible for construction may be preempted, allowing another thread to modify the head during this time. To address this, if the \texttt{compare\_exchange()} operation fails, the thread will loop and create a new head until it is successfully performed.

\begin{lstlisting}[language=C++, caption={Concurrent insertion of a lock into the head of a free-list.}, label={algorithm:concurrent_insert_block}]
void insert_block(BlockHeader blk) {
    Mapping mapping = get_mapping(blk.size);

    do {
        TaggedBlockHeader old_head = blocks[mapping].load();
        TaggedBlockHeader new_head = construct_head(old_head);
    } while(blocks[mapping].compare_exchange(old_head, new_head));

    update_bitmap(mapping);
}
\end{lstlisting}

The process of concurrently removing a block from a free-list, as outlined in Listing~\ref{algorithm:concurrent_remove_block}, is similarly defined to block insertion. If the \texttt{compare\_exchange()} operation fails when removing a block, there is no guarantee that the same free-list contains additional blocks. In cases where thread preemption occurs during removal, other threads might exhaust the free-list, which will make further operations fail. To solve this, \texttt{remove\_block()} either succeeds or fails without looping until successfully performing a swap, like \texttt{insert\_block()} does. Consequently, the caller must recalculate the mapping and call \texttt{remove\_block()} again, and thus loop outside the \texttt{remove\_block()} function.

\begin{lstlisting}[language=C++, caption={Concurrent removal of the head of a free-list.}, label={algorithm:concurrent_remove_block}]
BlockHeader *remove_block(Mapping mapping) {
    TaggedBlockHeader old_head = blocks[mapping];
    TaggedBlockHeader new_thead = construct_head(old_head);

    if(blocks[mapping].compare_exchange(old_head, new_head)) {
        update_bitmap(mapping);
        return new_head;
    }

    return nullptr;
}
\end{lstlisting}

%%% Local Variables:
%%% mode: latex
%%% TeX-master: "main"
%%% End:

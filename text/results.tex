
\subsection{Performance}
\label{sec:results:performance}

The result of the single allocation benchmark is presented in Figure~\ref{fig:allocation_performance}. Initializing the memory after allocating is observed to have minimal impact on performance immediately following an allocation request. Performing the absolute worst-case allocation, when only a larger block exists, is shown to be slower in all cases than when a suitable block exists. This is expected since more work is done than when a suitable block exists. Furthermore, the optimized version is on par with the performance of the reference version, while the general version is slower than the other two versions. 
\begin{figure}[h]
    \centering
    \includesvg[width=0.8\textwidth]{figures/allocation_performance.svg}
    \caption{Performance of a single allocation on the different versions of the allocator when a suitable block exists, a suitable block exists and initializing the memory immediately after, and only a larger block exists.}
    \label{fig:allocation_performance}
\end{figure}

In contrast, the benchmark result of the single-free benchmark is shown in Figure~\ref{fig:free_performance}. The general and optimized versions are observed to be comparable and are both about 25\% slower than the reference version. It is worth noting that both the reference and general version have coalescing disabled, which has an impact on performance and should be taken into account when considering the results.

\begin{figure}[h]
    \centering
    \includesvg[width=0.80\textwidth]{figures/free_performance.svg}
    \caption{Performance when freeing blocks of a single size. Immediate coalescing is disabled for the reference and general version, and is not implemented in the optimized version.}
    \label{fig:free_performance}
\end{figure}

The ratio between allocations and frees for the patterns collected from real-world programs is shown in Figure~\ref{fig:program_ratios}, which also includes the ratio of free calls made with \texttt{nullptr} as an argument, i.e., \texttt{free(nullptr)}. The reason this is highlighted is that calling free on a \texttt{nullptr} returns immediately without doing anything, which is a valid operation according to the \texttt{malloc()} API.

\begin{figure}[h]
    \centering
    \includesvg[width=0.80\textwidth]{figures/program_ratios.svg}
    \caption{Ratio between allocations and frees for the benchmark programs, also showing the ratio of frees performed on null pointers. The total number of allocations and frees are shown below the program name in parentheses.}
    \label{fig:program_ratios}
\end{figure}

Performance results of applying patterns of allocations and frees are shown in Figure~\ref{fig:program_benchmarks}, with absolute values in Table~\ref{table:program_benchmarks}. The reference version is observed to be about 12\% faster on average than the other two versions across all programs, while the general and optimized versions are on par in most programs, with slight deviations. There is an overhead of logistically applying the allocations and frees, which is added to all programs and is included in the results.

\begin{figure}[h]
    \centering
    \includesvg[width=0.80\textwidth]{figures/program_benchmark.svg}
    \caption{Run time when applying patterns from different programs. The graph is linearly scaled such that the general version is always at 1.0 for performance comparison.}
    \label{fig:program_benchmarks}
\end{figure}

\begin{table}[h]
    \centering
    \begin{tabular}{l|cccccc}
    {} & {cat} & {grep} & {ls} & {nano} & {sed} & {wc} \\
    \hline
    Reference & 599 & 5684 & 1028 & 111489 & 4466 & 685 \\
    General   & 730 & 6333 & 1202 & 122532 & 5018 & 832 \\
    Optimized & 713 & 6333 & 1181 & 120467 & 4980 & 822 \
    \end{tabular}
    \caption{Mean benchmark results when applying patterns from different programs. Results are given in microseconds.}
    \label{table:program_benchmarks}
\end{table}

\FloatBarrier

\subsection{Internal Fragmentation}

The sizes of the block headers for the different versions are 32 bytes, of which 16 bytes can be used by allocated blocks for the reference version; 32 bytes for the general version; and 0 bytes for the optimized version. 

In the worst case with maximum padding necessary, the reference version will waste up to 57.5\%, the general version 69.6\% and the optimized version 29.2\% if filling the heap with as many allocations as possible of 17 bytes. Looking at the second case where there is no padding necessary for alignment, the reference version will waste up to 50\%, the general version 66\% and the optimized version 0\%, if filling the heap with as many allocations as possible of 16 bytes.

To summarize, when filling the heap with as small allocations as possible and considering the size of the block header and any necessary padding due to alignment, the reference version will waste 50\%-57.5\%, the general version 66\%-69.6\% and the optimized version 0\%-29.2\%, depending on the amount of padding that is applied. This means that for the optimized version, the single source of internal fragmentation is due to the padding that is applied.

%%% Local Variables:
%%% mode: latex
%%% TeX-master: "main"
%%% End:

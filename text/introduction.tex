
Memory management is an important aspect of modern software development, particularly in Java applications where dynamic memory allocation and garbage collection play a central role. Java automatically manages memory through garbage collection, which allocates and frees memory without intervention by the programmer, allowing for more productive and robust software development. However, this convenience can be challenging to realize efficiently, especially concerning performance and scalability. 

In the Open Java Development Kit (OpenJDK), the Java Virtual Machine (JVM) provides a set of garbage collectors, one of which is the Z garbage collector (ZGC). 

ZGC divides the available memory into so-called pages that can be operated on concurrently. Memory is allocated sequentially on a per-page basis, using a method called bump-pointer allocation, which keep track of a position where allocations are made that is moved every new allocation. A notable downside of bump-pointer allocation is its inability to allocate anywhere on a page. This makes it hard to reuse memory that has been allocated previously without moving all living objects away to another page. 

This inherent inflexibility of bump-pointer allocation may contribute to increased intra-page fragmentation, as objects are allocated strictly at the top of available memory, potentially leaving unused space within the page. An alternative method is to use a free-list to keep track of where memory can be allocated, on a per-page level. Allocating using a free-list can potentially distribute objects more efficiently throughout the page, thereby reducing fragmentation and maximizing memory utilization.

There are cases where a free-list have been used in the context of garbage collection to allocate memory. For example, free-list have been used in the well-established mark-sweep algorithm where memory is not compacted and thus becomes fragmented over time. The advantage of a free-list is that it easily keeps track of fragmented memory and facilitates object allocation to such memory. Research on the implementation of free-list-based memory allocators with sophisticated garbage collections algorithms such as ZGC is currently limited. This project aims to explore potential challenges and opportunities in this area. Our results show that...

% This project intends to gain insight into what challenges and opportunities there are when using a free-list-based memory allocator in a more sophisticated garbage collection algorithm like ZGC.

% TODO: What specific insight do we contribute with?

%%% Local Variables:
%%% mode: latex
%%% TeX-master: "main"
%%% End:


Memory management is a crucial aspect of modern software development, particularly in Java applications where dynamic memory allocation and garbage collection play a central role. Java automatically allocates and frees memory through garbage collection, relieving programmers of this task and facilitating more productive and robust software development. However, realizing this convenience efficiently can be challenging, particularly in terms of performance and scalability.

In the Open Java Development Kit (OpenJDK), the Java Virtual Machine (JVM) provides a set of garbage collectors, one of which is the Z Garbage Collector (ZGC)~\cite{zgc}. ZGC divides the available memory into pages that can be operated on concurrently. Memory is allocated sequentially on a page using a method called bump-pointer allocation, which tracks the position where allocations are made and moves it forward with each new allocation. A notable downside of bump-pointer allocation is its inability to allocate memory anywhere within a page, making it difficult to reuse memory freed behind the bump-pointer without relocating all live objects to another page.

This inherent inflexibility of bump-pointer allocation may contribute to increased intra-page fragmentation, as objects are allocated at the top of available memory, potentially leaving unused space within the page. An alternative method is to use a free-list to keep track of where memory can be allocated on a per-page level. Allocating memory using a free-list can potentially distribute objects more efficiently throughout the page, thereby reducing fragmentation and maximizing memory utilization.

This inherent inflexibility of bump-pointer allocation may contribute to increased intra-page fragmentation, as objects are allocated at the top of available memory, potentially leaving unused space within the page. An alternative method is to use a free-list to keep track of where memory can be allocated on a per-page level, potentially reducing fragmentation and maximizing memory utilization.

The advantage of a free-list~\cite{gchandbook} is that it easily tracks fragmented memory and facilitates object allocation within such memory. There are instances where free-lists have been used in the context of garbage collection to allocate memory. For example, the well-established mark-sweep algorithm~\cite{gchandbook} does not compact memory and thus uses a free-list to manage the fragmentation that results from this approach. This project aims to explore potential challenges and opportunities in the area of customising a free-list-based memory allocator for use in ZGC.

%%% Local Variables:
%%% mode: latex
%%% TeX-master: "main"
%%% End:

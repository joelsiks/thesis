
\subsection{Integration With ZGC}
\label{sec:future-work:integration}

As highlighted thoroughly throughout the report, the natural next step of this work is to integrate it into ZGC. As mentioned briefly in Section~\ref{sec:individual_contrubitons}, N. Gärds~\cite{niclas_report} has, in parallel to when this work was performed, investigated integrating an allocator into ZGC. His work focuses on the challenges and opportunities of integrating any free-list allocator into ZGC, not specifically the optimized TLSF allocator described in this work. As such, it would be interesting to use the findings of N. Gärds to integrate the optimized TLSF allocator into ZGC so that its impact may be fully understood in the environment it is intended to be used.

\newpage

\subsection{Smaller Minimum Allocation Size}
\label{sec:future-work:lilliput}

The smallest possible unit of allocation inside ZGC is 16 bytes at the time of writing, which is limited by the size of the Java object header, which is always aligned to a minimum of 16 bytes on a 64-bit system. Work is ongoing to reduce the size of the object header to 8 bytes or less through project Lilliput~\cite{lilliput}. Since Lilliput is still considered work in progress, allocation sizes smaller than 16 bytes are not supported in the general nor optimized version of the allocator.

It would be impractical to use 8 bytes to store information about the size of such small objects if or when adding support for smaller allocations in the future, as is done in the design of the optimized allocator. As discussed in Section~\ref{sec:tlsf} regarding the design of TLSF and continuing in Section~\ref{sec:adaptations:architectural-considerations} on architectural adjustments, after aligning to 8 bytes instead of 4, there is an extra bit available for storing metadata inside block headers. The extra bit could be used to compact the header for smaller allocations even more, by using it as a binary property to indicate whether a block is the smallest possible size, 8 bytes, or not. If the bit is set, the size field in the block header can be replaced with the next and prev pointers, making it possible to store all metadata about the block inside 8 bytes.

\subsection{Addressing Starvation}
\label{sec:future-work:starvation}

One of the main selling points of TLSF is its bounded worst-case performance, which makes it an attractive choice in real-time systems where long response times are undesirable. The TLSF algorithm is able to achieve its bounded performance because it has no loops in its design. However, when introducing concurrency using a lock-free mechanism, as described in Section~\ref{sec:adaptations_impl:concurrency} this is no longer the case. The lock-free implementation of the free-lists in the optimized version requires adding a loop to make sure that the CAS instruction eventually succeeds. Thus, for concurrent use-cases, the worst-case performance of the optimized allocator is not bounded and may end up in cases where all but one thread starve. 

To solve the issue of starvation, a strategy worth exploring is combining the lock-free implementation with a wait-free one that guarantees progress for every thread, thereby preventing starvation. A. Kogan and E. Petrank~\cite{fast_wait_free} propose a way to do this by creating fast wait-free data structures, which can be applied to the lock-free solution that designed in this work. The idea is to execute the efficient lock-free version most of the time, and execute a slower wait-free version only when things go wrong. This would make sure that per-thread progress is guaranteed and that threads do not end up starving.

%%% Local Variables:
%%% mode: latex
%%% TeX-master: "main"
%%% End:


% What has been the method so far?

The methodology is divided into four parts:

\begin{enumerate}
    \item Implement a reference version of an allocator and verify its functionality.
    \item Identify important aspects for memory allocation in garbage collection.
    \item Adapt the reference version in regard to important aspects.
    \item Evaluate both the reference and adapted version of the allocator.
\end{enumerate}

The very first thing that was done was to implement a reference version of a memory allocator, TLSF in this case. The reference version will be tested using real-world programs to make sure that it works. The main method used to test and verify the correctness of the reference implementation is to replace \texttt{malloc()} and \texttt{free()} with a wrapper that uses our reference allocator. The wrapper redefines all memory allocation functions defined in C's \texttt{stdlib.h}\footnote{malloc.3\url{https://man7.org/linux/man-pages/man3/malloc.3.html}}. To override libc's provided \texttt{malloc} implementation, we will use Linux's \texttt{LD\_PRELOAD} environment variable, which allows us to preload shared libraries that will be used before any system libraries.

The next step is identifying what areas are most reasonable and likely to have the largest impact when adapting the allocator.

With a set of approaches for adapting the allocator in our disposal we will continue with incrementally adapt and test the allocator. Most adaptations will be tested through a set of independent unit tests that ensure the allocator works as intended. However, as detailed in Section~\ref{sec:delimitations}, the allocator will not be tested through integration with ZGC.


%%% Local Variables:
%%% mode: latex
%%% TeX-master: "main"
%%% End:

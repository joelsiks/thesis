
The methodology is divided into four distinct phases:

\begin{enumerate}
    \item Implement a reference version of an allocator and verify its functionality.
    \item Identify important aspects for memory allocation in garbage collection.
    \item Adapt the reference version with regard to aspects from previous step.
    \item Evaluate the adaptations and how they compare to the reference version.
\end{enumerate}

The initial step involves implementing a reference version of a memory allocator, specifically the TLSF allocator described in the previous section. Although any free-list based allocator could have been chosen for this work, TLSF stands out in the way that it aligns well a major goal in ZGC: to have a bounded and fast response time. The functionality of the reference implementation is verified using real-world programs. This verification process entails substituting the standard memory allocation functions like malloc/free with a wrapper that utilizes the reference allocator. The wrapper effectively overrides these functions as defined in libc~\cite{mallocman}. The method employed to override functions utilizes the \texttt{LD\_PRELOAD} environment variable in Linux, which enables the preloading of shared libraries prior to system libraries.

The next step is identifying significant aspects related to memory allocation within the context of garbage collection. This identification step is an important heuristic to determine areas where adaptations to the allocator are likely to yield substantial impact. Finding aspects will mainly be done through reviewing literature in the area of memory allocation and garbage collection as well as insights about ZGC and general memory usage patterns in Java programs.

Building upon the insights gained from the previous step, the reference allocator is adapted in areas which are likely to be beneficial for performance. This adaptation process is conducted incrementally, with each modification being tested through unit tests to ensure that the allocator behaves as intended in every step.

Finally, both the reference allocator and its adapted versions will be compared. More on specific metrics, evaluation method and steps to reproduce in Section~\ref{sec:evaluation}.

%evaluated in regard to fragmentation and performance. Fragmentation is an important metric to consider for memory utilization efficiency, memory overhead and long-term stability of the program. Performance is also a valuable metric since it directly impacts the responsiveness of the allocator, and is measured in processor cycles per operation (allocating and freeing memory).

%%% Local Variables:
%%% mode: latex
%%% TeX-master: "main"
%%% End:

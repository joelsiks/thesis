
To reiterate, the purpose of this work is to explore, implement, and evaluate adaptations to the free-list-based TLSF memory allocator for use in ZGC. To achieve this, a method divided into four distinct phases has been followed:

\begin{enumerate}
    \item Implement a reference version of an allocator and test its functionality to use as a starting point for adaptations.
    \item Identify important aspects of memory allocation in garbage collection to make impactful adaptations to the reference allocator.
    \item Adapt the reference version with regard to the aspects identified in the previous step.
    \item Evaluate both the reference version and adapted versions of the allocator in terms of fragmentation, memory efficiency, and runtime performance measured using wall clock time.
\end{enumerate}

The first phase involves implementing a reference version of an allocator and verifying its functionality. This step is crucial, as it provides a solid foundation for subsequent adaptations. The functionality of the reference implementation is tested using real-world programs by substituting \texttt{malloc()} and \texttt{free()} with a wrapper that utilizes the reference allocator. This substitution is done using the \texttt{LD\_PRELOAD} environment variable in Linux, enabling the preloading of shared libraries before system libraries, thus allowing the use of a different allocator. The specific programs used for verification include commonly used Linux programs: \texttt{cat}, \texttt{grep}, \texttt{ls}, \texttt{nano}, \texttt{sed}, and \texttt{wc}. These programs are selected for their diversity and frequent use, making them a reasonable compromise for ensuring allocator functionality. Although more extensive testing would be ideal, this subset provides a practical balance between thoroughness and resource boundaries.

The second phase focuses on identifying important aspects of memory allocation in the context of garbage collection. This step is essential for making impactful adaptations to the reference allocator. Key aspects are identified through a comprehensive literature review on memory allocation and garbage collection, as well as an analysis of ZGC and memory usage patterns in Java programs. Understanding these aspects provides insights into areas with significant potential for improvement, ensuring that the adaptations made in the next phase are well-informed and targeted. The findings from this step will answer the first research question on what constraints are present in the context of allocating memory in ZGC.

In the third phase, the reference allocator is adapted based on the findings from the previous step. Adaptations are carried out gradually, with each modification being tested through unit tests to verify that the allocator functions as intended. This incremental approach ensures that each change is deliberate and aligns with the goal of optimizing the allocator for use with ZGC. By systematically addressing the identified aspects, the adaptations aim to enhance memory efficiency and reduce fragmentation.

The final phase involves evaluating both the reference and adapted versions of the allocator in terms of memory efficiency and runtime performance, measured using wall clock time. The same set of Linux programs used for verification are employed in this evaluation. This comparative analysis answers the research question regarding how the constraints of ZGC can be leveraged to optimize the allocator. Detailed metrics, evaluation methods, and steps for reproducing the results are provided in Section~\ref{sec:evaluation}.

%%% Local Variables:
%%% mode: latex
%%% TeX-master: "main"
%%% End:


This section draws heavily from \textit{The Garbage Collection Handbook: The Art of Automatic Memory Management}~\cite{gchandbook}, by Jones et al., which provides a comprehensive overview of garbage collection methods, both historical and current.

As mentioned in Section~\ref{sec:memory_management}, garbage collection is a form of automatic memory management where the system identifies and cleans up unused objects, which are considered garbage. This removes the requirement of managing memory manually, which reduces the possibility of user-related memory issues occurring. However, this also removes some control from the user and could lead to an increased memory footprint.

From the perspective of the GC, the user program can be perceived solely as a memory mutator. This is because the GC is only concerned with the memory management of the program, not its functionality. Therefore, the threads of the user program are commonly referred to as mutator threads, or just mutators.

During GC execution, it may be necessary to temporarily halt the mutator threads to give the GC exclusive memory access. This is known as a stop-the-world approach. From the mutator's perspective, the GC appears to be atomic. This permits a simple implementation and avoids synchronization between GC and mutators while the GC is running, but also introduces pauses in the application which may be undesirable if latency is a quality metric. The GC could also operate concurrently and in parallel with the mutator threads. This leads to increased complexity as the memory can mutate during GC execution. Most GCs fall somewhere in between these ends, executing concurrently with the mutator threads while occasionally requiring pauses in their execution.

There exist numerous garbage collection strategies. One is the mark-sweep algorithm, which begins by identifying a set of accessible ``root'' objects. Roots are objects that serve as entry points into the object graph, such as global variables, local variables in stack frames, and registers. Roots are then marked as live and scanned to identify other objects in the object graph. This process continues recursively until all live objects have been identified. Afterwards, the heap is swept, meaning that the space not occupied by live objects is made available for new allocations.

The mark-sweep algorithm does not move objects, which can result in higher levels of external fragmentation. The mark-compact algorithm addresses this issue. Like mark-sweep, it traverses the heap and marks all live objects. However, instead of sweeping, live objects are moved and compacted. This approach reduces fragmentation and can be managed using a simple bump-pointer allocator. 

A copying garbage collector divides the heap into many subheaps. A subheap is first chosen and marked as non-empty. Allocations are then made sequentially onto this subheap using bump-pointer allocation. When garbage collection is initiated, live objects are copied from their current subheap to another subheap. Once all live objects have been copied, the original subheap is marked as empty and its bump-pointer reset. Compared to mark-compact, heap size is significantly reduced when using few subheaps. Using many subheaps will make each subheap small, which could result in more, and likely shorter, garbage collection cycles.

%%% Local Variables:
%%% mode: latex
%%% TeX-master: "main"
%%% End:

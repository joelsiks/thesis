
In this section, adaptations of the reference implementation of the TLSF allocator are described to make it more suitable for use in the context of a garbage collector, specifically ZGC. Each adaptation is justified with an explanation of its relevance and the specific improvements it aims to achieve. Adaptations that require a more detailed explanation regarding their implementation are further elaborated in Section~\ref{sec:adaptations-impl}.

\subsection{Architectural Considerations}
\label{sec:adaptations:architectural-considerations}

ZGC is implemented for 64-bit architecture and does not support 32-bit~\cite{zgc:deep_dive}. Hence, in an effort to adapt the allocator for ZGC, it is reasonable for it to also be limited to the same scope. TLSF is designed primarily for 32-bit systems as it targets real-time embedded systems, which usually operate on a 32-bit architecture. However, the design principles and concepts of TLSF are not inherently tied to any specific architecture. Converting TLSF from 32-bit to 64-bit mainly requires aligning memory to a word size of 8 bytes instead of 4, which adds a slightly larger memory overhead if allocations are not using the extra memory.

As explained in Section~\ref{sec:tlsf}, the two least significant bits within the block header are reserved for metadata. When aligning addresses to 8 bytes instead of 4 bytes, an extra bit can be used to store metadata. A possible use of this extra bit is discussed in Section~\ref{sec:future-work:lilliput}.

\subsection{General and Optimized Versions}

The two use cases that will be considered when further adapting the allocator are using it either as a normal allocator in cases where a traditional \texttt{malloc()} and \texttt{free()} combination is used or as a way to allocate objects on pages in ZGC. A way to realize this is by making the allocator configurable, similar to what E. Berger et al.~\cite{configurable_allocator} have done, utilizing C++ templates and inheritance. Having a configurable allocator is not only desirable from a maintenance perspective, but it also allows adaptation to new environments and requirements. Additionally, it becomes easier to compare different versions of the allocator by changing the configuration, streamlining the process of measuring the efficacy of any potential improvements.

Moving on, the use case of using the allocator as a general allocator is referred to as the general version, and the adapted version for ZGC as the optimized version. The general version will aim to be as similar to the reference implementation as possible.

\subsection{Reduced Allocation Size Range}
\label{sec:adaptations:reduced_allocation_range}

In ZGC, small pages limit the user to allocate in a specific size range, namely [16 B, 256 KB], as mentioned in Section~\ref{sec:zpage}. As this range is only a fraction of the range that TLSF supports, the number of free-lists that are needed, and in turn the number of bits needed in the bitmaps, can be significantly reduced.

14 first-levels are needed to index every power of two inside the small page range, with a lower bound of $2^4 =$ 16 B and an upper bound of $2^{17} =$ 128 KB. It is desirable to store all bitmap information in a single 64-bit value, so that efficient bit instructions can be used on lookup. The number of second-levels should thus be chosen so that the bitmap stays below the 64-bit limit. For efficiency reasons again, the number of second levels should be a power of two~\cite{TLSF} so that bit-shift instructions can be used. The highest value this leaves us is 4, which results in 56-bits for storing all necessary metadata.

Since a small page is 2 MB large, the initial block will be of the same size, which is significantly larger than the maximum allocation size of 256 KB. There are two ways to solve this: either break the initial block into smaller blocks or allow storing blocks larger than the maximum allocation size. For simplicity's sake, another free-list is added that stores blocks larger than the maximum allocation size, called a ``large-list'', at the 57th bit.

To summarize, the new bitmap uses 57 bits to store information for all free-lists, which fits inside a word of 64 bits. Having a single word allows finding free-lists that contain blocks faster, using a more simplified indexing process. Additionally, when considering concurrency later on, atomic operations can be used when performing updates on the bitmap. A potential drawback of reducing the number of second-levels to 4 is that it could negatively impact the granularity of block sizes and the number of splits that might be performed. This does not pose a major drawback in ZGC and for Java programs, where most allocations are small, as the highest granularity is found for small blocks.

\subsection{Transitioning Between Different Allocation Strategies}

Completely phasing out the utilization of bump-pointer allocation may not be optimal, especially when there is no constant requirement to allocate solely within the ``holes'' of a page. Specifically, because bump-pointer allocation is faster than using a free-list-based allocator. Instead, users could employ bump-pointer allocation up to a certain point, at which it becomes advantageous to transition to a more sophisticated allocator for non-bump-pointer allocation.

Most allocators start with an unused chunk of memory, allowing them to track and keep an up-to-date record of what the underlying memory contains. However, when transitioning from bump-pointer to an allocator, the underlying memory has contents that are not represented in the allocator. The bump-pointer only stores information about where the next allocation should occur, not the specific location of used and unused memory. To solve this, the allocator should provide a way to update its internal representation to align with the contents of the memory. In ZGC, the latest live analysis of a page contains information about what data is alive in the page's underlying memory. This information can therefore be used to construct free blocks that match the unused parts of the underlying memory.

Constructing the representation of the allocator is a two-step process. Initially, the free-lists are cleared to discard any previous data. When clearing the allocator, a block is allocated that covers the entirety of the underlying memory and the bitmap is reset. Then, blocks of free memory are iteratively inserted into the allocator based on information from the latest live analysis in ZGC. Insertion is done by providing the start address of the block and its size. This places all responsibility on the caller to make sure that it aligns with the actual data stored inside the page, which should align with the data from the live analysis.

\subsection{Deferred Coalescing of Blocks}

In the process of allocating and freeing blocks within an allocator, the general preference is to always have as large blocks as possible available to fulfill larger requests. However, in scenarios where small blocks are frequently allocated and freed, the emphasis on coalescing blocks to accommodate larger requests in the future may not be worthwhile. Instead, coalescing blocks will be deferred until a later point in time or not done at all.

The idea behind deferred coalescing is that when the user calls \texttt{free()}, the block being freed is not coalesced with its neighboring blocks. Instead, the allocator's API is extended to include an explicit coalesce operation. This operation scans all blocks and coalesces as many as possible during a single sweep, which can be an expensive operation.

The explicit coalesce operation iterates over all blocks, requiring only knowledge of the size of the current block to identify the location of the next block. This eliminates the need to store a pointer to the physical neighbor right before the current block, which is only used when coalescing in a \texttt{free()} call. The method is shown in Listing~\ref{algorithm:coalesce_blocks}.

\begin{lstlisting}[language=C++, caption={Method for explicitly coalescing all possible free blocks in the allocator. Note that coalesce\_neighbors() removes both blocks from the free-list before the newly coalesced block is inserted.}, label={algorithm:coalesce_blocks}]
void coalesce_blocks() {
    BlockHeader *current = first_physical_block;

    while(current != nullptr) {
        BlockHeader *next = current->next;

        if(next != nullptr && current->is_free() && next->is_free()) {
            current = coalesce_neighbors(current, next);
            insert_block(current);
        } else {
            current = next;
        }
    }
}
\end{lstlisting}

\subsection{Block Header Adjustments}
\label{sec:adaptations:block-header-adjustments}

According to R. Jones et al.~\cite[Page 103]{gchandbook}, when using allocators in garbage collectors, it is a reasonable approach to remove certain parts of the so-called boundary tag, or block header, for TLSF. As opposed to a general-purpose allocator designed for use in any system, an allocator used in a garbage collector does not need to keep track of as much information, as some of it is already kept track of in the GC itself.

The goal of adjusting the header is to reduce the constant block header overhead required when allocating blocks. The reduction is applied to headers for all allocations but is especially noteworthy for smaller blocks, where the header makes up a larger portion of the total memory used for that specific allocation. This aligns well with the fact that most allocations in Java programs are small, making the adjustment more significant.

From the list of adaptations discussed so far, it can be observed that storing a pointer to the previous physical block is unnecessary if the emphasis is shifted away from freeing and, in turn, coalescing blocks immediately. From this, it follows that the previous physical pointer can be removed from the block header in the optimized version to reduce the header overhead. Additionally, concerning the reduced allocation size range discussed in Section~\ref{sec:adaptations:reduced_allocation_range}, the footprint of the block header can be further reduced by converting the next and prev pointers to offsets instead. This would convert 64-bit pointers to 32-bit offsets, narrowing down the addressable space to 4 GB, which is more than enough to offset into ZGC's 2 MB pages.

Furthermore, storing the size of allocated blocks is redundant if it is also stored somewhere else, namely in the garbage collector itself, which ZGC does. This allows us to remove the size field completely for allocated blocks and only store it for free blocks. This change will require the user to provide the block size upon calling \texttt{free()} so that the block can be inserted into the appropriate free-list.

\subsection{Concurrency}

Considering that ZGC is a concurrent garbage collector, it is reasonable for the allocator to also support concurrency. Several potential use cases within ZGC could benefit from a concurrent allocator, such as thread-local garbage collection and reference counting in the old generation. However, at this stage, the allocator will not be adapted to any specific use case, apart from the general adaptation of it being concurrent. By supporting concurrency alongside other adaptations, the allocator becomes better equipped to remain relevant for evolving use cases beyond the scope of this thesis.

Previous adaptations, such as limiting the allocation size ranges, smaller bitmaps, and block header adjustments, significantly facilitate the implementation of concurrency in the allocator. As discussed in Section~\ref{sec:adaptations_impl:concurrency}, concurrency is supported using a lock-free approach. This requires addressing challenges such as the ABA problem.

%%% Local Variables:
%%% mode: latex
%%% TeX-master: "main"
%%% End:

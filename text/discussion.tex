
% Link results to research questions. Explain what the experiments tell you about your research questions. If any of the experiments “failed” or did not behave as ex- pected, try to reason about potential explanations and propose additional experiments that could be performed to lead the investigation further.

% Limitations and future work. Describe the limitations of your work, i.e., under what assumptions do your conclusions hold? What potentially relevant aspects were outside the scope of your study? Possibly, suggest avenues for future work building upon your study.

Exhaustive and comprehensive benchmarking of allocators is not feasible, and the performance results should not be interpreted as definitive. The performance benchmarks and results do not necessarily indicate how Java-programs will perform when using the allocator in ZGC. Applying patterns instead of using the allocators directly inside programs is also not optimal as it does not show how the allocators behave in practice. For example, without program logic mixed in between allocations and frees, the cache-locality is not fully representative, which might impact performance and thus the reliability of the results. With these considerations in mind, the benchmarks do however compare the allocators on a level which is fairly defined, allowing us to reason about their relative performance. 

From the results we can conclude that for single allocations, the optimized version is comparable in performance to the reference version, but is about 25\% slower than the reference version for single frees. When applying patterns from a selection of real-world programs, the optimized version is about 12\% slower on average. 

For programs with fewer total operations, the optimized version is observed to be slower than for programs with more total operations, even though the programs with more operations perform more frees, which should, looking at the performance of single frees, be slower overall. One reason for this might be cache-performance and memory re-use playing a larger role in programs that do more operations than fewer. With the benchmarks results being limited in mind, a hypothesis that can be made is that performing more total operations over time makes the performance difference between the reference and optimized versions less noticeable. This hypothesis will have to be further examined, but might be of interest when considering using the allocator in ZGC, where it could be desirable to use the allocator in scenarios where many operations are done.

The largest adaptation of the allocator is the concept of the 0-byte header. The 0-byte header stores no information inside allocated blocks, but uses the first 16 bytes of a free block to store metadata, and thus requires the minimum allocation size to be 16, which aligns well with the same limit that ZGC has. The benefit followed by the adaptation, in addition to using less memory, is that allocated memory can be packed more closely together, making more memory fit inside the same cache-line, increasing cache-locality, and likely performance as a result. This is a major benefit that is likely to yield better performance, and is also shared as a goal of the Lilliput project~\cite{lilliput} in the OpenJDK. Lilliput is discussed more, as future work, in Section~\ref{sec:future-work:lilliput}.

Looking at the worst-case for internal fragmentation, we can show that the only source of internal fragmentation for the optimized version is due to padding. Since the optimized version is able to completely remove the block header for allocated blocks, there is practically no memory overhead for allocated blocks apart from padding. This means that the optimized version is no different in terms of internal fragmentation from using bump-pointer allocation, which also applies padding to meet alignment requirements.

%%% Local Variables:
%%% mode: latex
%%% TeX-master: "main"
%%% End:

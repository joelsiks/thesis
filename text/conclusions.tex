
This thesis explored various adaptations to the TLSF memory allocator to enhance performance and memory efficiency within the ZGC garbage collector in OpenJDK. A reference version of TLSF was implemented, followed by a configurable version from which an optimized implementation was derived. These adaptations aimed to tailor TLSF for ZGC's specific requirements.

Understanding the way the allocator is used and also the environment in which it operates has opened up possibilities to either omit, alter, or add certain functionalities. For instance, ZGC, which already manages metadata about the objects it allocates, allows the allocator to disregard data that overlap with this. This adaptation, along with optimizations concerning the distribution and frequency of allocation sizes, has improved both performance and memory usage. Furthermore, adding support for concurrent operations on the allocator was significantly simplified by streamlining the representation and operations on the free-lists, allowing the optimized allocator to support concurrent calls to \texttt{malloc()} and \texttt{free()} through lock-free operations.

More in-depth evaluation is required to fully understand how the allocator behaves when used in ZGC. In isolation, the optimized allocator performs comparably to the reference implementation for single allocations but shows a slowdown for single frees and when applying patterns from real-world programs. The introduction of a 0-byte header for allocated blocks in the optimized version results in significantly smaller worst-case internal fragmentation, entirely eliminated when no padding is applied. In summary, the results show promise for future integration into ZGC, which would be a natural next step for this project. Integration will also provide a conclusive environment to evaluate and understand the impact of the adaptations that have been explored.

%%% Local Variables:
%%% mode: latex
%%% TeX-master: "main"
%%% End:


This thesis has explored multiple adaptations to the TLSF memory allocator to improve either performance or memory efficiency in the context of using it in ZGC, a garbage collector part of the OpenJDK. To do this, a reference version of TLSF has been implemented, from which a configurable version has been derived. Using the configurable version, an optimized implementation is explored with a set of adaptations aimed at tailoring TLSF for use in ZGC.

Understanding the way the allocator is used and also the environment in which it is used opens up possibilities to either omit, alter or add certain functionality. A garbage collector, like ZGC, already store and keep track of metadata regarding objects that it allocate, allowing the allocator to ignore parts that overlap with this information. Regarding both distribution and frequency of allocation sizes, internal representations of the allocator can be made more efficient in terms of both performance and memory usage. Adding support for concurrent operations on the allocator was made significantly easier by simplifying the representation and operations on the free-lists. The optimized allocator supports concurrent calls to \texttt{malloc()} and \texttt{free()} through lock-free operations on the free-lists.

More in-depth evaluation is required to fully understand how the allocator behaves when used in ZGC. In isolation, the allocator performs on par with the reference implementation when performing single allocations, but is slower when performing single frees. Applying allocation patterns from a set of real-world programs indicate how the allocators behave for those programs which is not necessarily overlapping with Java programs. As a result of the 0-byte header for allocated blocks in the optimized version, the worst-case internal fragmentation is significantly smaller compared to the reference version, and is completely removed when no padding is applied. In summary, the results show promise for future integration into ZGC, which is the natural next step of the work done in this project. Integration will also provide a more conclusive environment to evaluate and fully understand the impact of the adaptations that have been explored.

%%% Local Variables:
%%% mode: latex
%%% TeX-master: "main"
%%% End:


This project has explored multiple possible adaptations that can be made to the TLSF memory allocator to improve either performance or memory efficiency in the context of using it in ZGC, a garbage collector part of the OpenJDK. To do this, a reference version of TLSF has been implemented, from which a configurable version has been implemented. From configuration, an optimized version for ZGC that implements the explored adaptations is implemented.

Understanding the way the allocator is used and also the environment in which it is used opens up possibilities to either omit, alter or add several parts it. A garbage collector like ZGC already store and keep track of metadata regarding objects that it allocate, allowing the allocator to ignore parts that overlap with this and are redundant to manage. Regarding both distribution and frequency of allocation sizes, internal representations can be made more efficient in terms of both performance and memory usage, when reasoned about theoretically.

More in-depth evaluation is required to fully understand how the allocator behaves when used in ZGC. In isolation, the allocator performs on par with the reference implementation when performing single allocations. As a result of the 0-byte header for allocated blocks in the optimized version, the worst-case internal fragmentation is significantly smaller compared to the reference version, and is completely removed when no padding is applied. In summary, the results show promise for future integration into ZGC, which is the natural next step of the work done in this project. Integration will also provide a more conclusive environment to evaluate and fully understand the impact of the adaptations that have been made.

%%% Local Variables:
%%% mode: latex
%%% TeX-master: "main"
%%% End:

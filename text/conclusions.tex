
This project has explored multiple possible adaptations that can be made to the TLSF memory allocator to improve either performance or memory efficiency in the context of using it in ZGC, a garbage collector part of the OpenJDK. To do this, a reference version of TLSF has been implemented, which makes it easy to both implement new adaptations and to compare them to an appropriate baseline.

Understanding the way the allocator is used and also the environment in which it is used opens up possibilities to either omit, alter or add several parts it. A garbage collector like ZGC already store and keep track of metadata regarding objects that it allocate, allowing the allocator to ignore parts that overlap with this and thus is redundant to manage. Regarding both distribution and frequency of allocation sizes, internal representations can be made more efficient in terms of both performance and memory usage.

Concurrency is something that is both desirable to have and needed in some use-cases for ZGC. However, adding support for this removes one of the main selling points of TLSF, namely that it no longer has a bounded worst-case performance due to possible contention and preemption.

More in-depth evaluation is required to fully understand how the allocator behaves when used in ZGC. In isolation, the allocator performs on par with the reference implementation when performing single allocations. As a result of the 0-byte header for allocated blocks in the optimized version, the internal fragmentation is significantly smaller compared to the reference version, and is completely removed when no padding is applied. In summary, the results show great promise for future integration into ZGC, which is the natural next step of the work done in this project.

%%% Local Variables:
%%% mode: latex
%%% TeX-master: "main"
%%% End:
